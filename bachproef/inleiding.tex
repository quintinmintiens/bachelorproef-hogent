%%=============================================================================
%% Inleiding
%%=============================================================================

\chapter{\IfLanguageName{dutch}{Inleiding}{Introduction}}%
\label{ch:inleiding}

De warmwalserij bij ArcelorMittal, waar metalen platen worden verwarmd en gewalst, is een energie-intensief onderdeel van het staalproductieproces. Gezien de aanzienlijke hoeveelheden energie die dit proces vereist, is efficiënt energiebeheer van cruciaal belang. Eén van de grootste uitdagingen hierbij is het tijdig detecteren van anomalieën in het energieverbruik van de ovens, wat essentieel is voor het verminderen van verspilling en het optimaliseren van de operationele efficiëntie. Zo kan bijvoorbeeld een openstaande ovenklep leiden tot substantiële energieverliezen. Het tijdig identificeren van dergelijke anomalieën stelt ons in staat om corrigerende maatregelen te nemen en zo de efficiëntie van de productieprocessen te verhogen.

\section{Probleemstelling}
Deze studie adresseert het gebrek aan efficiënte methoden voor het detecteren van energieverbruiksanomalieën in de ovens van ArcelorMittal. Het ontbreken van effectieve detectiemethoden kan resulteren in onnodig energieverlies en operationele inefficiënties, wat de kosten verhoogt en de duurzaamheid van het productieproces ondermijnt.

\section{Onderzoeksvraag}
De centrale onderzoeksvraag van deze studie luidt: "Hoe kunnen we een model ontwikkelen dat in staat is om het energieverbruik van de ovens van ArcelorMittal nauwkeurig te voorspellen en anomalieën te detecteren?"

\section{Onderzoeksdoelstelling}
Het primaire doel van dit onderzoek is het ontwikkelen van een betrouwbaar model dat niet alleen het energieverbruik van de ovens nauwkeurig kan voorspellen, maar ook in staat is om significante afwijkingen te identificeren die duiden op potentiële inefficiënties of defecten in het productieproces.

\section{Opzet van deze Bachelorproef}
In Hoofdstuk~\ref{ch:stand-van-zaken} wordt een uitgebreide literatuurstudie gepresenteerd die de bestaande kennis en technieken binnen het domein van energiemanagement en anomaliedetectie middels machine learning onderzoekt. Dit hoofdstuk identificeert ook de leemtes in de huidige onderzoekslandschap die deze studie beoogt te adresseren.

In Hoofdstuk~\ref{ch:methodologie} wordt de methodologie gedetailleerd beschreven die gebruikt is voor het verzamelen en verwerken van de data, het selecteren van de kenmerken, het ontwikkelen van het voorspellingsmodel, en het evalueren van dit model. Er wordt ook toegelicht hoe de dataset is opgebouwd en welke criteria zijn gebruikt om de modelarchitectuur te kiezen.

In Hoofdstuk~\ref{ch:conclusie}, tenslotte, worden de belangrijkste conclusies van de studie samengevat en wordt er een antwoord gegeven op de onderzoeksvraag. Dit hoofdstuk biedt ook aanbevelingen voor toekomstig onderzoek en praktische implementaties van de studieresultaten.
