%%=============================================================================
%% Samenvatting
%%=============================================================================

% TODO: De "abstract" of samenvatting is een kernachtige (~ 1 blz. voor een
% thesis) synthese van het document.
%
% Een goede abstract biedt een kernachtig antwoord op volgende vragen:
%
% 1. Waarover gaat de bachelorproef?
% 2. Waarom heb je er over geschreven?
% 3. Hoe heb je het onderzoek uitgevoerd?
% 4. Wat waren de resultaten? Wat blijkt uit je onderzoek?
% 5. Wat betekenen je resultaten? Wat is de relevantie voor het werkveld?
%
% Daarom bestaat een abstract uit volgende componenten:
%
% - inleiding + kaderen thema
% - probleemstelling
% - (centrale) onderzoeksvraag
% - onderzoeksdoelstelling
% - methodologie
% - resultaten (beperk tot de belangrijkste, relevant voor de onderzoeksvraag)
% - conclusies, aanbevelingen, beperkingen
%
% LET OP! Een samenvatting is GEEN voorwoord!

%%---------- Nederlandse samenvatting -----------------------------------------
%
% TODO: Als je je bachelorproef in het Engels schrijft, moet je eerst een
% Nederlandse samenvatting invoegen. Haal daarvoor onderstaande code uit
% commentaar.
% Wie zijn bachelorproef in het Nederlands schrijft, kan dit negeren, de inhoud
% wordt niet in het document ingevoegd.

\IfLanguageName{english}{%
\selectlanguage{dutch}
\chapter*{Samenvatting}
\lipsum[1-4]
\selectlanguage{english}
}{}

%%---------- Samenvatting -----------------------------------------------------
% De samenvatting in de hoofdtaal van het document

\chapter*{\IfLanguageName{dutch}{Samenvatting}{Abstract}}


Dit onderzoek concentreert zich op het ontwikkelen van geavanceerde energiemodellen voor het detecteren van anomalieën in het energieverbruik van de ovens binnen de warmwalserij van ArcelorMittal. Deze studie is gemotiveerd door de kritieke noodzaak om energieverlies te minimaliseren, zoals in gevallen waarbij een ovenklep onbedoeld open blijft staan, wat leidt tot aanzienlijke inefficiënties.

In het kader van dit onderzoek zijn verschillende machine learning-modellen ontwikkeld, getraind en geëvalueerd met gebruikmaking van historische data over het energieverbruik van de ovens. De verzamelde dataset omvat diverse kenmerken zoals de gemiddelde breedte en dikte van de geproduceerde slabs, het energieverbruik van de ovens zowel tijdens productie als in rust, en het totale gewicht van de slabs per oven.

Uit een reeks experimenten met diverse machine learning-algoritmen bleek een Long Short-Term Memory (LSTM) netwerk het meest effectief. Dit LSTM-model, bestaande uit drie lagen en geoptimaliseerd met een sequence length van 64, behaalde een validation loss van 0.023, wat duidt op een hoge voorspellingsnauwkeurigheid van het energieverbruik.

De resultaten van het onderzoek illustreren dat slimme energiemodellen, gebaseerd op machine learning, succesvol anomalieën in het energieverbruik kunnen detecteren binnen industriële processen zoals die bij ArcelorMittal. Dit is van groot belang voor de industrie omdat het direct bijdraagt aan het optimaliseren van het energieverbruik, het verlagen van de energiekosten, en het ondersteunen van duurzaamheidsinitiatieven. Deze bevindingen bieden waardevolle inzichten en tools voor energiebeheer die potentieel brede toepassing kunnen vinden in vergelijkbare industriële omgevingen.

