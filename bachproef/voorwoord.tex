%%=============================================================================
%% Voorwoord
%%=============================================================================

\chapter*{\IfLanguageName{dutch}{Woord vooraf}{Preface}}%
\label{ch:voorwoord}

%% TODO:
%% Het voorwoord is het enige deel van de bachelorproef waar je vanuit je
%% eigen standpunt (``ik-vorm'') mag schrijven. Je kan hier bv. motiveren
%% waarom jij het onderwerp wil bespreken.
%% Vergeet ook niet te bedanken wie je geholpen/gesteund/... heeft


Met veel enthousiasme grijp ik deze gelegenheid aan om mijn voorwoord te wijden aan de aanvang van deze bachelorproef. Gedurende mijn studie heb ik een diepgaande passie ontwikkeld voor het verkennen van nieuwe technologieën en het begrijpen van hun toepassingen in real-world industriële contexten. Het thema van deze proef, het ontwikkelen van intelligente energiemodellen voor het detecteren van anomalieën in het energieverbruik bij ArcelorMittal, trok mijn aandacht om meerdere redenen.

Allereerst bood dit onderwerp een uitgelezen kans om de kracht van machine learning en deep learning toe te passen binnen een industriële omgeving, waar de efficiëntie van energieverbruik van essentieel belang is. Door geavanceerde modellen te ontwikkelen die in staat zijn anomalieën in energieverbruik te identificeren, streven we niet alleen naar het verlagen van operationele kosten, maar dragen we ook bij aan een duurzamere toekomst.

Daarnaast bood dit project een waardevolle gelegenheid om hands-on ervaring op te doen binnen een toonaangevende organisatie zoals ArcelorMittal. De samenwerking met experts uit de industrie en het aangaan van praktische uitdagingen hebben mijn professionele vaardigheden aanzienlijk verrijkt en mijn begrip van complexe industriële processen verdiept.

Bij deze wil ik mijn oprechte dank uitspreken aan iedereen die heeft bijgedragen aan dit onderzoek. Speciale dank gaat uit naar:
\begin{itemize}
    \item Giselle Vercauteren, mijn bachelorproefbegeleider, voor haar onmisbare begeleiding en continue ondersteuning doorheen dit onderzoekstraject.
    \item Teun de Raad en Alexander Vande Ghinste, mijn co-promotors en stagementors, voor hun deskundige adviezen en bemoedigende steun gedurende mijn stage bij ArcelorMittal.
\end{itemize}
Zonder hun inzichten, steun en expertise zou dit project niet tot stand zijn gekomen.

Ik hoop dat deze bachelorproef niet alleen een waardevolle bijdrage levert aan mijn academische ontwikkeling, maar ook een constructieve bijdrage zal leveren aan het veld van energie-efficiëntie en industriële automatisering. Het is mijn oprechte wens dat de resultaten en inzichten die in dit werk worden gepresenteerd anderen zullen inspireren en bijdragen aan een efficiëntere en duurzamere toekomst.

