%===============================================================================
% LaTeX sjabloon voor de bachelorproef toegepaste informatica aan HOGENT
% Meer info op https://github.com/HoGentTIN/latex-hogent-report
%===============================================================================

\documentclass[dutch,dit,thesis]{hogentreport}

% TODO:
% - If necessary, replace the option `dit`' with your own department!
%   Valid entries are dbo, dbt, dgz, dit, dlo, dog, dsa, soa
% - If you write your thesis in English (remark: only possible after getting
%   explicit approval!), remove the option "dutch," or replace with "english".

\usepackage{lipsum} % For blind text, can be removed after adding actual content

%% Pictures to include in the text can be put in the graphics/ folder
\graphicspath{{graphics/}}

%% For source code highlighting, requires pygments to be installed
%% Compile with the -shell-escape flag!
\usepackage[section]{minted}
%% If you compile with the make_thesis.{bat,sh} script, use the following
%% import instead:
%% \usepackage[section,outputdir=../output]{minted}
\usemintedstyle{solarized-light}
\definecolor{bg}{RGB}{253,246,227} %% Set the background color of the codeframe

%% Change this line to edit the line numbering style:
\renewcommand{\theFancyVerbLine}{\ttfamily\scriptsize\arabic{FancyVerbLine}}

%% Macro definition to load external java source files with \javacode{filename}:
\newmintedfile[javacode]{java}{
    bgcolor=bg,
    fontfamily=tt,
    linenos=true,
    numberblanklines=true,
    numbersep=5pt,
    gobble=0,
    framesep=2mm,
    funcnamehighlighting=true,
    tabsize=4,
    obeytabs=false,
    breaklines=true,
    mathescape=false
    samepage=false,
    showspaces=false,
    showtabs =false,
    texcl=false,
}

% Other packages not already included can be imported here

%%---------- Document metadata -------------------------------------------------
% TODO: Replace this with your own information
\author{Quintin Mintiens}
\supervisor{Mevr. G. Vercauteren}
\cosupervisor{Dhr. T. de Raad, Dhr. A. Vande Ghinste}
\title[]%
    {Slimme Energiemodellen in de Warmwalserij bij ArcelorMittal}
\academicyear{\advance\year by -1 \the\year--\advance\year by 1 \the\year}
\examperiod{1}
\degreesought{\IfLanguageName{dutch}{Professionele bachelor in de toegepaste informatica}{Bachelor of applied computer science}}
\partialthesis{false} %% To display 'in partial fulfilment'
%\institution{Internshipcompany BVBA.}

%% Add global exceptions to the hyphenation here
\hyphenation{back-slash}

%% The bibliography (style and settings are  found in hogentthesis.cls)
\usepackage[style=apa,backend=biber]{biblatex}
\addbibresource{bachproef.bib}            %% Bibliography file
\addbibresource{../voorstel/voorstel.bib} %% Bibliography research proposal
\defbibheading{bibempty}{}

%% Prevent empty pages for right-handed chapter starts in twoside mode
\renewcommand{\cleardoublepage}{\clearpage}

\renewcommand{\arraystretch}{1.2}

%% Content starts here.
\begin{document}

%---------- Front matter -------------------------------------------------------

\frontmatter

\hypersetup{pageanchor=false} %% Disable page numbering references
%% Render a Dutch outer title page if the main language is English
\IfLanguageName{english}{%
    %% If necessary, information can be changed here
    \degreesought{Professionele Bachelor toegepaste informatica}%
    \begin{otherlanguage}{dutch}%
       \maketitle%
    \end{otherlanguage}%
}{}

%% Generates title page content
\maketitle
\hypersetup{pageanchor=true}

%%=============================================================================
%% Voorwoord
%%=============================================================================

\chapter*{\IfLanguageName{dutch}{Woord vooraf}{Preface}}%
\label{ch:voorwoord}

%% TODO:
%% Het voorwoord is het enige deel van de bachelorproef waar je vanuit je
%% eigen standpunt (``ik-vorm'') mag schrijven. Je kan hier bv. motiveren
%% waarom jij het onderwerp wil bespreken.
%% Vergeet ook niet te bedanken wie je geholpen/gesteund/... heeft


Met veel enthousiasme grijp ik deze gelegenheid aan om mijn voorwoord te wijden aan de aanvang van deze bachelorproef. Gedurende mijn studie heb ik een diepgaande passie ontwikkeld voor het verkennen van nieuwe technologieën en het begrijpen van hun toepassingen in real-world industriële contexten. Het thema van deze proef, het ontwikkelen van intelligente energiemodellen voor het detecteren van anomalieën in het energieverbruik bij ArcelorMittal, trok mijn aandacht om meerdere redenen.

Allereerst bood dit onderwerp een uitgelezen kans om de kracht van machine learning en deep learning toe te passen binnen een industriële omgeving, waar de efficiëntie van energieverbruik van essentieel belang is. Door geavanceerde modellen te ontwikkelen die in staat zijn anomalieën in energieverbruik te identificeren, streven we niet alleen naar het verlagen van operationele kosten, maar dragen we ook bij aan een duurzamere toekomst.

Daarnaast bood dit project een waardevolle gelegenheid om hands-on ervaring op te doen binnen een toonaangevende organisatie zoals ArcelorMittal. De samenwerking met experts uit de industrie en het aangaan van praktische uitdagingen hebben mijn professionele vaardigheden aanzienlijk verrijkt en mijn begrip van complexe industriële processen verdiept.

Bij deze wil ik mijn oprechte dank uitspreken aan iedereen die heeft bijgedragen aan dit onderzoek. Speciale dank gaat uit naar:
\begin{itemize}
    \item Giselle Vercauteren, mijn bachelorproefbegeleider, voor haar onmisbare begeleiding en continue ondersteuning doorheen dit onderzoekstraject.
    \item Teun de Raad en Alexander Vande Ghinste, mijn co-promotors en stagementors, voor hun deskundige adviezen en bemoedigende steun gedurende mijn stage bij ArcelorMittal.
\end{itemize}
Zonder hun inzichten, steun en expertise zou dit project niet tot stand zijn gekomen.

Ik hoop dat deze bachelorproef niet alleen een waardevolle bijdrage levert aan mijn academische ontwikkeling, maar ook een constructieve bijdrage zal leveren aan het veld van energie-efficiëntie en industriële automatisering. Het is mijn oprechte wens dat de resultaten en inzichten die in dit werk worden gepresenteerd anderen zullen inspireren en bijdragen aan een efficiëntere en duurzamere toekomst.


%%=============================================================================
%% Samenvatting
%%=============================================================================

% TODO: De "abstract" of samenvatting is een kernachtige (~ 1 blz. voor een
% thesis) synthese van het document.
%
% Een goede abstract biedt een kernachtig antwoord op volgende vragen:
%
% 1. Waarover gaat de bachelorproef?
% 2. Waarom heb je er over geschreven?
% 3. Hoe heb je het onderzoek uitgevoerd?
% 4. Wat waren de resultaten? Wat blijkt uit je onderzoek?
% 5. Wat betekenen je resultaten? Wat is de relevantie voor het werkveld?
%
% Daarom bestaat een abstract uit volgende componenten:
%
% - inleiding + kaderen thema
% - probleemstelling
% - (centrale) onderzoeksvraag
% - onderzoeksdoelstelling
% - methodologie
% - resultaten (beperk tot de belangrijkste, relevant voor de onderzoeksvraag)
% - conclusies, aanbevelingen, beperkingen
%
% LET OP! Een samenvatting is GEEN voorwoord!

%%---------- Nederlandse samenvatting -----------------------------------------
%
% TODO: Als je je bachelorproef in het Engels schrijft, moet je eerst een
% Nederlandse samenvatting invoegen. Haal daarvoor onderstaande code uit
% commentaar.
% Wie zijn bachelorproef in het Nederlands schrijft, kan dit negeren, de inhoud
% wordt niet in het document ingevoegd.

\IfLanguageName{english}{%
\selectlanguage{dutch}
\chapter*{Samenvatting}
\lipsum[1-4]
\selectlanguage{english}
}{}

%%---------- Samenvatting -----------------------------------------------------
% De samenvatting in de hoofdtaal van het document

\chapter*{\IfLanguageName{dutch}{Samenvatting}{Abstract}}


Dit onderzoek concentreert zich op het ontwikkelen van geavanceerde energiemodellen voor het detecteren van anomalieën in het energieverbruik van de ovens binnen de warmwalserij van ArcelorMittal. Deze studie is gemotiveerd door de kritieke noodzaak om energieverlies te minimaliseren, zoals in gevallen waarbij een ovenklep onbedoeld open blijft staan, wat leidt tot aanzienlijke inefficiënties.

In het kader van dit onderzoek zijn verschillende machine learning-modellen ontwikkeld, getraind en geëvalueerd met gebruikmaking van historische data over het energieverbruik van de ovens. De verzamelde dataset omvat diverse kenmerken zoals de gemiddelde breedte en dikte van de geproduceerde slabs, het energieverbruik van de ovens zowel tijdens productie als in rust, en het totale gewicht van de slabs per oven.

Uit een reeks experimenten met diverse machine learning-algoritmen bleek een Long Short-Term Memory (LSTM) netwerk het meest effectief. Dit LSTM-model, bestaande uit drie lagen en geoptimaliseerd met een sequence length van 64, behaalde een validation loss van 0.023, wat duidt op een hoge voorspellingsnauwkeurigheid van het energieverbruik.

De resultaten van het onderzoek illustreren dat slimme energiemodellen, gebaseerd op machine learning, succesvol anomalieën in het energieverbruik kunnen detecteren binnen industriële processen zoals die bij ArcelorMittal. Dit is van groot belang voor de industrie omdat het direct bijdraagt aan het optimaliseren van het energieverbruik, het verlagen van de energiekosten, en het ondersteunen van duurzaamheidsinitiatieven. Deze bevindingen bieden waardevolle inzichten en tools voor energiebeheer die potentieel brede toepassing kunnen vinden in vergelijkbare industriële omgevingen.



%---------- Inhoud, lijst figuren, ... -----------------------------------------

\tableofcontents

% In a list of figures, the complete caption will be included. To prevent this,
% ALWAYS add a short description in the caption!
%
%  \caption[short description]{elaborate description}
%
% If you do, only the short description will be used in the list of figures

\listoffigures

% If you included tables and/or source code listings, uncomment the appropriate
% lines.
%\listoftables
%\listoflistings

% Als je een lijst van afkortingen of termen wil toevoegen, dan hoort die
% hier thuis. Gebruik bijvoorbeeld de ``glossaries'' package.
% https://www.overleaf.com/learn/latex/Glossaries

%---------- Kern ---------------------------------------------------------------

\mainmatter{}

% De eerste hoofdstukken van een bachelorproef zijn meestal een inleiding op
% het onderwerp, literatuurstudie en verantwoording methodologie.
% Aarzel niet om een meer beschrijvende titel aan deze hoofdstukken te geven of
% om bijvoorbeeld de inleiding en/of stand van zaken over meerdere hoofdstukken
% te verspreiden!

%%=============================================================================
%% Inleiding
%%=============================================================================

\chapter{\IfLanguageName{dutch}{Inleiding}{Introduction}}%
\label{ch:inleiding}

De inleiding moet de lezer net genoeg informatie verschaffen om het onderwerp te begrijpen en in te zien waarom de onderzoeksvraag de moeite waard is om te onderzoeken. In de inleiding ga je literatuurverwijzingen beperken, zodat de tekst vlot leesbaar blijft. Je kan de inleiding verder onderverdelen in secties als dit de tekst verduidelijkt. Zaken die aan bod kunnen komen in de inleiding~\autocite{Pollefliet2011}:

\begin{itemize}
  \item context, achtergrond
  \item afbakenen van het onderwerp
  \item verantwoording van het onderwerp, methodologie
  \item probleemstelling
  \item onderzoeksdoelstelling
  \item onderzoeksvraag
  \item \ldots
\end{itemize}

\section{\IfLanguageName{dutch}{Probleemstelling}{Problem Statement}}%
\label{sec:probleemstelling}

Uit je probleemstelling moet duidelijk zijn dat je onderzoek een meerwaarde heeft voor een concrete doelgroep. De doelgroep moet goed gedefinieerd en afgelijnd zijn. Doelgroepen als ``bedrijven,'' ``KMO's'', systeembeheerders, enz.~zijn nog te vaag. Als je een lijstje kan maken van de personen/organisaties die een meerwaarde zullen vinden in deze bachelorproef (dit is eigenlijk je steekproefkader), dan is dat een indicatie dat de doelgroep goed gedefinieerd is. Dit kan een enkel bedrijf zijn of zelfs één persoon (je co-promotor/opdrachtgever).

\section{\IfLanguageName{dutch}{Onderzoeksvraag}{Research question}}%
\label{sec:onderzoeksvraag}

Wees zo concreet mogelijk bij het formuleren van je onderzoeksvraag. Een onderzoeksvraag is trouwens iets waar nog niemand op dit moment een antwoord heeft (voor zover je kan nagaan). Het opzoeken van bestaande informatie (bv. ``welke tools bestaan er voor deze toepassing?'') is dus geen onderzoeksvraag. Je kan de onderzoeksvraag verder specifiëren in deelvragen. Bv.~als je onderzoek gaat over performantiemetingen, dan 

\section{\IfLanguageName{dutch}{Onderzoeksdoelstelling}{Research objective}}%
\label{sec:onderzoeksdoelstelling}

Wat is het beoogde resultaat van je bachelorproef? Wat zijn de criteria voor succes? Beschrijf die zo concreet mogelijk. Gaat het bv.\ om een proof-of-concept, een prototype, een verslag met aanbevelingen, een vergelijkende studie, enz.

\section{\IfLanguageName{dutch}{Opzet van deze bachelorproef}{Structure of this bachelor thesis}}%
\label{sec:opzet-bachelorproef}

% Het is gebruikelijk aan het einde van de inleiding een overzicht te
% geven van de opbouw van de rest van de tekst. Deze sectie bevat al een aanzet
% die je kan aanvullen/aanpassen in functie van je eigen tekst.

De rest van deze bachelorproef is als volgt opgebouwd:

In Hoofdstuk~\ref{ch:stand-van-zaken} wordt een overzicht gegeven van de stand van zaken binnen het onderzoeksdomein, op basis van een literatuurstudie.

In Hoofdstuk~\ref{ch:methodologie} wordt de methodologie toegelicht en worden de gebruikte onderzoekstechnieken besproken om een antwoord te kunnen formuleren op de onderzoeksvragen.

% TODO: Vul hier aan voor je eigen hoofstukken, één of twee zinnen per hoofdstuk

In Hoofdstuk~\ref{ch:conclusie}, tenslotte, wordt de conclusie gegeven en een antwoord geformuleerd op de onderzoeksvragen. Daarbij wordt ook een aanzet gegeven voor toekomstig onderzoek binnen dit domein.
\chapter{\IfLanguageName{dutch}{Stand van zaken}{State of the art}}%
\label{ch:stand-van-zaken}

\section{Inleiding}

In de hedendaagse wereld staat energiebeheer centraal in de strategische planning van zowel overheden als private sectoren. Dit komt voornamelijk door de toenemende wereldbevolking, industrialisatie en de urgentie van klimaatverandering, die allemaal bijdragen aan de complexiteit van energievoorziening en -consumptie. De International Energy Agency (IEA) rapporteert dat de wereldwijde energievraag naar verwachting met meer dan 25\% zal toenemen tegen 2040, tenzij er effectieve beleidsmaatregelen worden geïmplementeerd \autocite{Reddy_2023}.

Deze groeiende vraag naar energie gaat hand in hand met de noodzaak om de energie-efficiëntie te verhogen en de milieu-impact van energieproductie en -consumptie te verminderen. Technologieën zoals machine learning (ML) en deep learning (DL) bieden nieuwe perspectieven voor het voorspellen van energieverbruik, wat cruciaal is voor het optimaliseren van het energiebeheer en het minimaliseren van verspilling. Voorspellingsmodellen die gebaseerd zijn op deze technologieën kunnen helpen bij het nauwkeurig inschatten van energiebehoeften, waardoor overheden en bedrijven beter geïnformeerde beslissingen kunnen nemen over energieproductie, distributie en consumptie.

Machine learning en deep learning zijn bijzonder effectief in het modelleren van complexe, dynamische systemen waar traditionele statistische methoden tekortschieten. Deze methoden zijn in staat om grote hoeveelheden historische en real-time data te verwerken om patronen te identificeren die niet direct waarneembaar zijn. Het gebruik van deze geavanceerde analytische technieken in energievoorspellingsmodellen maakt een meer geïntegreerde en adaptieve benadering van energiebeheer mogelijk, wat van cruciaal belang is in een tijdperk waarin energiebronnen steeds meer onder druk staan.

Echter, de implementatie van ML en DL in energievoorspelling is niet zonder uitdagingen. Deze technologieën vereisen grote hoeveelheden data van goede kwaliteit, geavanceerde algoritmen voor gegevensverwerking en -analyse, en aanzienlijke rekenkracht. Daarnaast is de interpretatie van de resultaten van ML- en DL-modellen vaak complex en vereist het diepgaande technische kennis, wat een belemmering kan vormen voor organisaties zonder gespecialiseerde expertise.

Ondanks deze uitdagingen bieden de potentiële voordelen van het toepassen van machine learning en deep learning in energievoorspelling aanzienlijke kansen voor zowel de publieke als de private sector. Door energieverbruik nauwkeuriger te voorspellen, kunnen organisaties niet alleen hun operationele efficiëntie verbeteren, maar ook bijdragen aan duurzamere energiepraktijken die essentieel zijn voor het milieu.

In de volgende secties zullen we dieper ingaan op specifieke ML- en DL-technieken die worden gebruikt voor energievoorspelling, inclusief hun toepassingen, prestaties en de uitdagingen die met hun implementatie gepaard gaan.

\section{Machine Learning Technieken voor Energievoorspelling}

\subsection{Overzicht van Machine Learning Technieken}
Machine learning technieken hebben een breed scala aan toepassingen in energievoorspelling. Deze technieken omvatten regressieanalyses, beslisbomen, support vector machines (SVM), en neurale netwerken, elk met zijn eigen sterktes afhankelijk van de aard en complexiteit van de voorspellingsopdracht.

\begin{enumerate}
    \item Regressieanalyse en Tijdreeksmodellen: Zoals uiteengezet in de studies door \textcite{Reddy_2023}, worden regressieanalyses vaak gebruikt voor het voorspellen van energieverbruik op basis van historische data. Dit paper illustreert de toepassing van meerdere regressiemodellen om energieverbruik te voorspellen met een hoge mate van nauwkeurigheid, waarbij tijdreeksanalyses worden benut om seizoensgebonden trends in elektriciteitsvraag te modelleren.
    \item Beslisbomen en Random Forests: Random Forest algoritmen worden breed toegepast in de energievoorspelling, zoals besproken door \textcite{Reddy_2023}. Deze algoritmen zijn bijzonder nuttig vanwege hun vermogen om grote datasets te hanteren en complexe beslissingsbomen op te bouwen die helpen bij het classificeren en voorspellen van energieverbruikspatronen.
    \item Support Vector Machines (SVM): SVMs worden gebruikt voor zowel classificatie als regressie in energievoorspellingsscenario's. De studie door \textcite{Reddy_2023} toont aan hoe SVM ingezet kan worden om niet-lineaire relaties efficiënt te modelleren en complexe patronen in energiegegevens te identificeren, wat essentieel is voor het nauwkeurig voorspellen van piekvraag en -verbruik.
    \item Neurale Netwerken: De toepassing van geavanceerde neurale netwerken, zoals gedemonstreerd in \textcite{Alanbar_2020}, benadrukt de kracht van diepe leerarchitecturen zoals Long Short-Term Memory (LSTM) netwerken voor het voorspellen van energieverbruik in educatieve gebouwen. Deze technieken zijn bijzonder waardevol vanwege hun vermogen om lange termijn afhankelijkheden in tijdreeksdata vast te leggen.
\end{enumerate}

\subsection{Toepassing van ML-technieken in Energievoorspelling}
In de praktijk zijn deze machine learning technieken cruciaal voor het ontwikkelen van energiebeheersystemen die realtime data kunnen analyseren en voorspellingen kunnen doen. Het gebruik van ML in de energiesector helpt niet alleen bij het voorspellen van de vraag en het aanbod om de efficiëntie te verhogen, maar ook bij het identificeren van anomalieën in energieverbruik, wat cruciaal is voor het onderhouden van de netwerkintegriteit en het verminderen van verliezen.

De door \textcite{Reddy_2023} gepresenteerde case studies illustreren hoe machine learning kan worden ingezet om de operationele efficiëntie te verbeteren en energiekosten te verlagen door nauwkeurigere voorspellingen van energiebehoeften, wat cruciaal is voor de planning en operatie van moderne energienetwerken.

\section{Deep Learning voor Geavanceerde Energievoorspelling}

Deep learning, een geavanceerde vorm van machine learning, speelt een steeds grotere rol in de voorspelling van energieverbruik door zijn vermogen om complexe patronen en relaties in grote datasets te herkennen. Deze sectie onderzoekt hoe deep learning technieken, met name neurale netwerken zoals LSTM, worden toegepast voor nauwkeurige energievoorspellingen.

\subsection{Long Short-Term Memory Netwerken (LSTM)}
LSTM-netwerken zijn een type recurrent neural network (RNN) speciaal ontworpen om het probleem van het verdwijnen van de gradient, een veelvoorkomend probleem bij standaard RNNs, op te lossen. Door hun architectuur zijn LSTMs in staat om informatie voor lange perioden te behouden, wat ze ideaal maakt voor tijdreeksvoorspelling zoals het voorspellen van energieverbruik.

In een studie uitgevoerd aan de Qassim Universiteit, zoals beschreven door \textcite{Alanbar_2020}, werd LSTM gebruikt om het energieverbruik in een onderwijsinstelling te voorspellen. Deze studie toonde aan dat het LSTM-model een lage Root Mean Square Error (RMSE) bereikte, wat wijst op een hoge nauwkeurigheid vergeleken met relatieve studies. Deze resultaten benadrukken het potentieel van LSTM-netwerken voor het verbeteren van de voorspellingsnauwkeurigheid in real-time energiemanagementsystemen.

\subsection{Toepassingen van Deep Learning in Energievoorspelling}
Deep learning wordt niet alleen gebruikt voor het voorspellen van het energieverbruik maar ook voor het optimaliseren van energieproductie en -distributieprocessen. In de energiesector kunnen deep learning modellen bijvoorbeeld worden gebruikt om vraagresponsstrategieën te optimaliseren, waarbij de voorspelde energievraag wordt gebruikt om de energieproductie dynamisch aan te passen om efficiëntie te verhogen en kosten te verlagen.

Verder is deep learning bijzonder effectief in het identificeren van niet-lineaire relaties en patronen die moeilijk te detecteren zijn met traditionele statistische methoden. Dit maakt het een krachtig hulpmiddel voor scenario's waar complexe variabelen zoals weerpatronen, economische activiteiten en gebruikersgedrag het energieverbruik beïnvloeden.

\subsection{Uitdagingen en Overwegingen}

Hoewel deep learning aanzienlijke voordelen biedt, zijn er ook uitdagingen zoals de noodzaak van grote hoeveelheden gelabelde trainingsdata, hoge computationele vereisten en de complexiteit van modelafstemming en -validatie. Deze uitdagingen vereisen aanzienlijke resources en expertise, wat kan leiden tot beperkingen in de toepasbaarheid, vooral in kleinere organisaties of in ontwikkelingslanden waar toegang tot dergelijke resources beperkt kan zijn.

De case study door \textcite{Alanbar_2020} biedt een praktijkvoorbeeld van hoe deze uitdagingen kunnen worden aangepakt en hoe deep learning succesvol kan worden geïmplementeerd voor energievoorspelling in een academische setting. Echter, het is belangrijk om te erkennen dat er verschillen zijn tussen academische en industriële settings. In een academische omgeving zijn de doelen vaak gericht op onderzoek en innovatie met toegang tot gespecialiseerde kennis en middelen, terwijl industriële toepassingen vaak gericht zijn op praktische en schaalbare oplossingen binnen striktere budgetten en tijdslijnen.

Het parallel trekken van deze settings kan risico’s met zich meebrengen, zoals het overschatten van de beschikbare middelen en expertise in een industriële omgeving, of het onderschatten van de noodzaak voor rigoureuze validatie en schaalbaarheid van oplossingen. Het is daarom cruciaal om de contextuele verschillen in overweging te nemen bij het extrapoleren van bevindingen uit academische studies naar industriële toepassingen.

\section{Uitdagingen en Toekomstige Richtingen}

\subsection{Uitdagingen bij het Implementeren van ML en DL in Energievoorspelling}
Het toepassen van machine learning (ML) en deep learning (DL) technologieën op het gebied van energievoorspelling brengt verschillende technische en operationele uitdagingen met zich mee. Deze uitdagingen moeten worden aangepakt om de betrouwbaarheid en effectiviteit van voorspellingsmodellen te garanderen.

\begin{enumerate}
    \item Datakwaliteit en -volume: Zoals benadrukt in de studie van \textcite{Alanbar_2020}, vereisen ML en DL modellen grote hoeveelheden hoogwaardige data om nauwkeurige voorspellingen te genereren. Het verzamelen van voldoende data van hoge kwaliteit blijft een uitdaging, vooral in sectoren waar dataverzameling niet systematisch is geïntegreerd in de operationele processen.
    \item Modelcomplexiteit en interpretatie: De complexiteit van ML en DL modellen kan leiden tot moeilijkheden bij de interpretatie van de resultaten. Dit wordt verder gecompliceerd door het 'black box'-karakter van vele deep learning modellen, waarbij het niet altijd duidelijk is hoe beslissingen worden genomen. Deze kwestie van modeltransparantie is cruciaal, vooral in sectoren zoals de energievoorziening waar betrouwbaarheid en vertrouwen van groot belang zijn.
    \item Rekenintensiteit: De training van geavanceerde DL modellen vereist aanzienlijke rekenkracht, wat kan resulteren in hoge kosten en energieverbruik op zich. Dit aspect werd duidelijk in de toepassing van LSTM-netwerken voor energievoorspelling in de case study gepresenteerd door \textcite{Alanbar_2020}, waarbij aanzienlijke computationele middelen nodig waren voor het verwerken van de data.
\end{enumerate}

\subsection{Toekomstige Richtingen in Energievoorspelling}
Gezien de snelle ontwikkelingen op het gebied van ML en DL en de toenemende beschikbaarheid van data, zijn er veelbelovende richtingen voor toekomstig onderzoek en implementatie in energievoorspelling.

\begin{enumerate}
    \item Verbetering van de toegankelijkheid van technologie: Het ontwikkelen van meer gebruikersvriendelijke ML- en DL-tools die minder technische expertise vereisen, kan helpen om deze technologieën toegankelijker te maken voor een breder publiek. Dit zou de adoptie van geavanceerde voorspellingsmodellen in minder technisch georiënteerde industrieën kunnen versnellen.
    \item Integratie met IoT-apparaten: De integratie van ML en DL modellen met Internet of Things (IoT) apparaten in slimme netwerken biedt een veelbelovend pad voor real-time energiemanagement en -optimalisatie. Dit zou leiden tot meer gedecentraliseerde en efficiënte energiebeheersystemen.
    \item Focus op duurzaamheid: Verder onderzoek naar hoe ML en DL kunnen bijdragen aan duurzamere energiepraktijken is essentieel. Dit omvat het optimaliseren van het energieverbruik en het verminderen van de ecologische voetafdruk van energieproductie en -consumptie.
\end{enumerate}

\section{Conclusie}

De toepassing van machine learning (ML) en deep learning (DL) technologieën in de voorspelling van energieverbruik biedt aanzienlijke mogelijkheden voor de energie-industrie. Deze technieken bieden geavanceerde analytische mogelijkheden die traditionele modellen niet kunnen bieden, zoals het vermogen om complexe, niet-lineaire patronen en relaties binnen grote datasets te identificeren en te modelleren.

Hoewel de voordelen voor de energie-industrie duidelijk zijn, zijn er ook parallellen te trekken naar de bredere industriële sector. Zowel in de energie- als de industriële sector kan het gebruik van ML en DL leiden tot verbeterde operationele efficiëntie, nauwkeurigere voorspellingen en geoptimaliseerde resource-allocatie. Bijvoorbeeld, in de productie-industrie kunnen deze technologieën worden ingezet voor voorspellend onderhoud, kwaliteitscontrole en supply chain optimalisatie, net zoals ze in de energie-industrie worden gebruikt voor vraagvoorspelling en optimalisatie van energieverbruik.

Echter, er zijn ook belangrijke verschillen om rekening mee te houden. De specifieke uitdagingen en vereisten van een industriële omgeving, zoals de variëteit aan processen en machines, kunnen verschillen van die in de energie-industr

\subsection{Samenvatting van Kernpunten}
\begin{enumerate}
    \item Verbeterde nauwkeurigheid van voorspellingen: Zoals gedemonstreerd in de studies van \textcite{Alanbar_2020}, maken ML- en DL-technieken, met name LSTM-netwerken, het mogelijk om met hoge nauwkeurigheid energieverbruik te voorspellen, wat essentieel is voor het efficiënt beheren van energiebronnen en het verminderen van operationele kosten.
    \item Uitdagingen en barrières: Ondanks hun potentieel, komen de implementatie en exploitatie van ML- en DL-modellen met uitdagingen, waaronder de noodzaak voor grote hoeveelheden kwalitatieve data, hoge computationele eisen, en de complexiteit van modelinterpretatie en -validatie.
    \item Toekomstige richtingen: De voortdurende ontwikkeling van technologieën en methoden voor het verbeteren van de toegankelijkheid en duurzaamheid van ML- en DL-toepassingen zal cruciaal zijn. Dit omvat onderzoek naar hoe deze technieken geïntegreerd kunnen worden met opkomende technologieën zoals IoT voor real-time monitoring en controle, en hoe ze kunnen bijdragen aan duurzamere energiepraktijken.
\end{enumerate}

\subsection{Implicaties voor de Energiesector}
De toepassing van ML en DL in energievoorspelling heeft niet alleen de potentie om de operationele efficiëntie te verbeteren, maar biedt ook kansen voor het bevorderen van duurzame energie-initiatieven. Door een beter begrip van verbruikspatronen kunnen energiebedrijven beter reageren op de vraag, hun productie optimaliseren, en tegelijkertijd hun ecologische voetafdruk verminderen.

Bovendien, zoals de resultaten van de door \textcite{Alanbar_2020} uitgevoerde studie aantonen, kan de implementatie van deze technologieën helpen bij het ontwikkelen van meer geavanceerde voorspellingsmodellen die niet alleen economische voordelen bieden maar ook bijdragen aan het behoud van energie en het verlagen van emissies.

%%=============================================================================
%% Methodologie
%%=============================================================================

\chapter{\IfLanguageName{dutch}{Methodologie}{Methodology}}%
\label{ch:methodologie}

%% TODO: In dit hoofstuk geef je een korte toelichting over hoe je te werk bent
%% gegaan. Verdeel je onderzoek in grote fasen, en licht in elke fase toe wat
%% de doelstelling was, welke deliverables daar uit gekomen zijn, en welke
%% onderzoeksmethoden je daarbij toegepast hebt. Verantwoord waarom je
%% op deze manier te werk gegaan bent.
%% 
%% Voorbeelden van zulke fasen zijn: literatuurstudie, opstellen van een
%% requirements-analyse, opstellen long-list (bij vergelijkende studie),
%% selectie van geschikte tools (bij vergelijkende studie, "short-list"),
%% opzetten testopstelling/PoC, uitvoeren testen en verzamelen
%% van resultaten, analyse van resultaten, ...
%%
%% !!!!! LET OP !!!!!
%%
%% Het is uitdrukkelijk NIET de bedoeling dat je het grootste deel van de corpus
%% van je bachelorproef in dit hoofstuk verwerkt! Dit hoofdstuk is eerder een
%% kort overzicht van je plan van aanpak.
%%
%% Maak voor elke fase (behalve het literatuuronderzoek) een NIEUW HOOFDSTUK aan
%% en geef het een gepaste titel.

\lipsum[21-25]



% Voeg hier je eigen hoofdstukken toe die de ``corpus'' van je bachelorproef
% vormen. De structuur en titels hangen af van je eigen onderzoek. Je kan bv.
% elke fase in je onderzoek in een apart hoofdstuk bespreken.

%\input{...}
%\input{...}
%...

%%=============================================================================
%% Conclusie
%%=============================================================================

\chapter{Conclusie}%
\label{ch:conclusie}

% TODO: Trek een duidelijke conclusie, in de vorm van een antwoord op de
% onderzoeksvra(a)g(en). Wat was jouw bijdrage aan het onderzoeksdomein en
% hoe biedt dit meerwaarde aan het vakgebied/doelgroep? 
% Reflecteer kritisch over het resultaat. In Engelse teksten wordt deze sectie
% ``Discussion'' genoemd. Had je deze uitkomst verwacht? Zijn er zaken die nog
% niet duidelijk zijn?
% Heeft het onderzoek geleid tot nieuwe vragen die uitnodigen tot verder 
%onderzoek?

\lipsum[76-80]



%---------- Bijlagen -----------------------------------------------------------

\appendix

\chapter{Onderzoeksvoorstel}

Het onderwerp van deze bachelorproef is gebaseerd op een onderzoeksvoorstel dat vooraf werd beoordeeld door de promotor. Dat voorstel is opgenomen in deze bijlage.

%% TODO: 
%\section*{Samenvatting}

% Kopieer en plak hier de samenvatting (abstract) van je onderzoeksvoorstel.

% Verwijzing naar het bestand met de inhoud van het onderzoeksvoorstel
%---------- Introductie ---------------------------------------------------------
\section{Introductie}%
\label{sec:introductie}

Deze bachelorproef richt zich op het ontwikkelen van betrouwbare energiemodellen voor de warmbandwalserij bij ArcelorMittal. Deze modellen zullen worden gebruikt om energieverbruik te voorspellen en te vergelijken met het werkelijke energieverbruik. Verschillen tussen voorspeld en gemeten energie kunnen anomalieën in het proces aan het licht brengen.

\begin{itemize}
  \item Thema: Integratie van energiemodellen met procesdata voor verbeterde efficiëntie en anomaliedetectie.
  \item Doelgroep: Bachelor- en masterstudenten in computerwetenschappen of elektromechanica met interesse in datawetenschap, specifiek gericht op energiebeheer en optimalisatie van industriële processen bij ArcelorMittal.
  \item Probleemstelling: Huidige energiemodellen in de warmbandwalserij functioneren onafhankelijk van het productieproces, wat leidt tot onvoldoende detectie van anomalieën.
  \item Onderzoeksvraag: Hoe kunnen we energiemodellen verbeteren door ze te integreren met procesdata, zodat ze nauwkeuriger energieverbruik kunnen voorspellen en anomalieën beter kunnen detecteren?
  \item Onderzoeksdoelstelling: Ontwikkeling van geavanceerde energiemodellen die synergie hebben met procesdata voor betere voorspelling, monitoring en optimalisatie van energieverbruik.
\end{itemize}

%---------- State-of-the-art ---------------------------------------------------
\section{State-of-the-art}%
\label{sec:state-of-the-art}

In de huidige stand van zaken binnen het domein van energiemodellering in industriële omgevingen is er een toenemende integratie van big data en machine learning technieken om energieverbruik te voorspellen en te optimaliseren. Recente ontwikkelingen hebben zich gericht op het combineren van deze modellen met realtime productiegegevens om de nauwkeurigheid van de voorspellingen en de detectie van anomalieën te verbeteren. Er bestaan echter nog lacunes in de aanpassing van deze modellen aan specifieke industriële omgevingen, zoals de warmbandwalserij bij ArcelorMittal.

%---------- Methodologie ------------------------------------------------------
\section{Methodologie}%
\label{sec:methodologie}

De methodologie omvat:
\begin{itemize}
  \item Dataverzameling: Verzamelen van energieverbruiksgegevens en procesgegevens van de warmbandwalserij van ArcelorMittal.
  \item Modelontwikkeling: Gebruik van Python en machine learning technieken om energiemodellen te creëren.
  \item Integratie: Combineren van de energiemodellen met procesgegevens.
  \item Testen en Optimalisatie: Verfijnen van de modellen om de nauwkeurigheid van de voorspellingen en anomaliedetectie te verbeteren.
\end{itemize}
Het onderzoek zal voornamelijk een combinatie zijn van datawetenschap en machine learning toepassingen in een industriële omgeving.

%---------- Verwachte resultaten ----------------------------------------------
\section{Verwacht resultaat, conclusie}%
\label{sec:verwachte_resultaten}

De verwachte resultaten omvatten:
\begin{itemize}
  \item Een uitgebreid energiemodel dat nauwkeurig energieverbruik voorspelt in correlatie met het productieproces.
  \item Verbeterde detectie van anomalieën in het energieverbruik.
  \item Aanbevelingen voor energieoptimalisatie in de warmbandwalserij.
\end{itemize}
Deze resultaten zullen bijdragen aan een efficiënter energiebeheer in de warmbandwalserij van ArcelorMittal.

\end{document}

%%---------- Andere bijlagen --------------------------------------------------
% TODO: Voeg hier eventuele andere bijlagen toe. Bv. als je deze BP voor de
% tweede keer indient, een overzicht van de verbeteringen t.o.v. het origineel.
%\input{...}

%%---------- Backmatter, referentielijst ---------------------------------------

\backmatter{}

\setlength\bibitemsep{2pt} %% Add Some space between the bibliograpy entries
\printbibliography[heading=bibintoc]

\end{document}
