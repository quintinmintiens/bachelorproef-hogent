%==============================================================================
% Sjabloon onderzoeksvoorstel bachproef
%==============================================================================
% Gebaseerd op document class `hogent-article'
% zie <https://github.com/HoGentTIN/latex-hogent-article>

% Voor een voorstel in het Engels: voeg de documentclass-optie [english] toe.
% Let op: kan enkel na toestemming van de bachelorproefcoördinator!
\documentclass{hogent-article}

% Invoegen bibliografiebestand
\addbibresource{voorstel.bib}

% Informatie over de opleiding, het vak en soort opdracht
\studyprogramme{Professionele bachelor toegepaste informatica}
\course{Bachelorproef}
\assignmenttype{Onderzoeksvoorstel}
% Voor een voorstel in het Engels, haal de volgende 3 regels uit commentaar
% \studyprogramme{Bachelor of applied information technology}
% \course{Bachelor thesis}
% \assignmenttype{Research proposal}

\academicyear{2023-2024} % TODO: pas het academiejaar aan

% TODO: Werktitel
\title{Bachelorproef Voorstel: Slimme Energiemodellen in de Warmwalserij bij ArcelorMittal}

% TODO: Studentnaam en emailadres invullen
\author{Quintin Mintiens}
\email{quintin.mintiens@student.hogent.be}

% TODO: Medestudent
% Gaat het om een bachelorproef in samenwerking met een student in een andere
% opleiding? Geef dan de naam en emailadres hier
% \author{Yasmine Alaoui (naam opleiding)}
% \email{yasmine.alaoui@student.hogent.be}

% TODO: Geef de co-promotor op
\supervisor[Co-promotor]{}

% Binnen welke specialisatierichting uit 3TI situeert dit onderzoek zich?
% Kies uit deze lijst:
%
% - Mobile \& Enterprise development
% - AI \& Data Engineering
% - Functional \& Business Analysis
% - System \& Network Administrator
% - Mainframe Expert
% - Als het onderzoek niet past binnen een van deze domeinen specifieer je deze
%   zelf
%
\specialisation{AI \& Data Engineering}
\keywords{
  Energiemodellering, Procesoptimalisatie in de Industrie, Machine Learning, Anomaliedetectie, Duurzame Productie.}

\begin{document}

\begin{abstract}
  
In deze bachelorproef wordt een geïntegreerd energiemodel ontwikkeld voor de warmwalserij bij ArcelorMittal, gericht op het verbeteren van energie-efficiëntie en het detecteren van anomalieën in het energieverbruik. Het onderzoek verbindt bestaande energiemodellen met real-time procesdata door middel van datawetenschap en machine learning. De kernvraag is hoe deze integratie bijdraagt aan effectievere energiebeheerstrategieën en procesoptimalisatie. Het verwachte resultaat is een geavanceerd energiemodel dat nauwkeurige voorspellingen levert, wat van grote waarde is voor IT-professionals en technische studenten in de industriële sector. De methodologie omvat datacollectie, modelontwikkeling met Python en machine learning, en optimalisatie voor praktische toepassing. Dit onderzoek beoogt een significante bijdrage te leveren aan duurzamere productieprocessen in de staalsector.
\end{abstract}

\tableofcontents

% De hoofdtekst van het voorstel zit in een apart bestand, zodat het makkelijk
% kan opgenomen worden in de bijlagen van de bachelorproef zelf.
%---------- Introductie ---------------------------------------------------------
\section{Introductie}%
\label{sec:introductie}

Deze bachelorproef richt zich op het ontwikkelen van betrouwbare energiemodellen voor de warmbandwalserij bij ArcelorMittal. Deze modellen zullen worden gebruikt om energieverbruik te voorspellen en te vergelijken met het werkelijke energieverbruik. Verschillen tussen voorspeld en gemeten energie kunnen anomalieën in het proces aan het licht brengen.

\begin{itemize}
  \item Thema: Integratie van energiemodellen met procesdata voor verbeterde efficiëntie en anomaliedetectie.
  \item Doelgroep: Bachelor- en masterstudenten in computerwetenschappen of elektromechanica met interesse in datawetenschap, specifiek gericht op energiebeheer en optimalisatie van industriële processen bij ArcelorMittal.
  \item Probleemstelling: Huidige energiemodellen in de warmbandwalserij functioneren onafhankelijk van het productieproces, wat leidt tot onvoldoende detectie van anomalieën.
  \item Onderzoeksvraag: Hoe kunnen we energiemodellen verbeteren door ze te integreren met procesdata, zodat ze nauwkeuriger energieverbruik kunnen voorspellen en anomalieën beter kunnen detecteren?
  \item Onderzoeksdoelstelling: Ontwikkeling van geavanceerde energiemodellen die synergie hebben met procesdata voor betere voorspelling, monitoring en optimalisatie van energieverbruik.
\end{itemize}

%---------- State-of-the-art ---------------------------------------------------
\section{State-of-the-art}%
\label{sec:state-of-the-art}

In de huidige stand van zaken binnen het domein van energiemodellering in industriële omgevingen is er een toenemende integratie van big data en machine learning technieken om energieverbruik te voorspellen en te optimaliseren. Recente ontwikkelingen hebben zich gericht op het combineren van deze modellen met realtime productiegegevens om de nauwkeurigheid van de voorspellingen en de detectie van anomalieën te verbeteren. Er bestaan echter nog lacunes in de aanpassing van deze modellen aan specifieke industriële omgevingen, zoals de warmbandwalserij bij ArcelorMittal.

%---------- Methodologie ------------------------------------------------------
\section{Methodologie}%
\label{sec:methodologie}

De methodologie omvat:
\begin{itemize}
  \item Dataverzameling: Verzamelen van energieverbruiksgegevens en procesgegevens van de warmbandwalserij van ArcelorMittal.
  \item Modelontwikkeling: Gebruik van Python en machine learning technieken om energiemodellen te creëren.
  \item Integratie: Combineren van de energiemodellen met procesgegevens.
  \item Testen en Optimalisatie: Verfijnen van de modellen om de nauwkeurigheid van de voorspellingen en anomaliedetectie te verbeteren.
\end{itemize}
Het onderzoek zal voornamelijk een combinatie zijn van datawetenschap en machine learning toepassingen in een industriële omgeving.

%---------- Verwachte resultaten ----------------------------------------------
\section{Verwacht resultaat, conclusie}%
\label{sec:verwachte_resultaten}

De verwachte resultaten omvatten:
\begin{itemize}
  \item Een uitgebreid energiemodel dat nauwkeurig energieverbruik voorspelt in correlatie met het productieproces.
  \item Verbeterde detectie van anomalieën in het energieverbruik.
  \item Aanbevelingen voor energieoptimalisatie in de warmbandwalserij.
\end{itemize}
Deze resultaten zullen bijdragen aan een efficiënter energiebeheer in de warmbandwalserij van ArcelorMittal.

\end{document}

\printbibliography[heading=bibintoc]

\end{document}